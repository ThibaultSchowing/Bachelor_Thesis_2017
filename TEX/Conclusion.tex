

\chapter{Conclusion}




Ce projet de recherche a permis de mettre au jour des relations entre le café et le climat sans pour autant parvenir à des résultats de prédiction ou de caractérisation satisfaisants. Des données supplémentaires et du temps de recherche seraient nécessaires afin de pouvoir améliorer les résultats obtenus et d'obtenir d'autres informations utiles. \\

\noindent Les recherches effectuées ont eu un impact positif sur la FNC qui, en la personne de M. Felipe Rincón, vient d'engager un processus de normalisation de la récolte des données afin d'améliorer les possibilités d'analyses dans un future proche. \\

\noindent La relation entre la mauvaise qualité des grains et les fortes précipitations pourrait permettre de mettre en place des procédures particulières en cas de fortes pluies afin d'éviter des pertes inutiles. Une étude devrait cependant être faite afin d'identifier les facteurs reliant la pluie aux défauts et ainsi pouvoir luter contre, ces facteurs pouvant être naturels, humains ou logistiques. \\


\noindent Une étude réalisée précédemment\cite{Cenicafe} appuie nos résultats en mettant en avant le fait que les données environnementales de la région sont propices à la production de cafés de haute qualités et que les principaux défauts du cafés viennent surtout des étapes qui interviennent après la récolte. Ainsi, d'après cette étude, les meilleurs cafés proviennent, entre autres, de grains qui ont été traités selon certaines bonnes pratiques (Good Manufacturing Practices), traités par fermentation pour enlever le mucilage et séchés au soleil. \\


\noindent Les objectifs du cahier des charges ont été atteint dans la mesure où les données ont pu être analysées et les résultats interprétés. Cette expérience a été très enrichissante tant au niveau professionnel que personnel. Il s'agissait d'un cas réel d'étude avec des données et des analyses du terrain, pour le terrain. 



