

\chapter{Conclusion}
%DRAFT
%Qu'est-ce que le travail peut apporter, que faudrait-il faire pour qu'il apporte plus (données manquantes, autres méthodes à explorer) 

%Proposer un processus de normalisation

%Taste wheel: http://dailycoffeenews.com/2014/01/06/the-new-tasters-flavor-wheel-a-recalibration-of-coffee-dialogue/


%Proactif - communication verticale atypique (si sylvain abscent etc)

% Proposition à Cenicafé, processus de collecte de données et d'uniformisation

% Améliorer l'utilisation des méthodes -> parameter tunning

% Données manquantes: rendement, date de récolte (début et fin), qualité du sol !!!

% Interpréter PCA: on peut dire siun café a peu de chance d'etre très très bon


Ce projet de recherche a permis de mettre au jour des relations entre le café et le climat sans pour autant parvenir à des résultats de prédiction ou de caractérisation satisfaisants. Des données supplémentaires et du temps de recherche seraient nécessaires afin de pouvoir améliorer les résultats obtenus. \\

\noindent Les recherches effectuées ont eu un impact positif sur la FNC qui, en la personne de M Felipe Rincón, viennent d'engager un processus de normalisation de la récolte des données afin d'améliorer les possibilités d'analyses dans un future proche. \\

\noindent La relation entre la mauvaise qualité des grains et les fortes précipitations pourrait permettre de mettre en place des procédures particulières en cas de fortes pluies afin d'éviter des pertes inutiles. Une étude devrait cependant être faite afin d'identifier les éventuels facteurs reliant la pluie aux défauts et ainsi pouvoir luter contre, ces facteurs pouvant être naturels, humains ou logistiques. \\

\noindent Les objectifs du cahier des charges ont été atteint dans la mesure où les données ont pu être analysées et les résultats interprétés. Cette expérience a été très enrichissante tant au niveau professionnel que personnel. Il s'agissait d'un cas réel d'étude avec des données et des analyses du terrain, pour le terrain. 



