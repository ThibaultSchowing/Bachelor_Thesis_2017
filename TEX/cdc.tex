\renewcommand{\abstractname}{Cahier des Charges}
\begin{abstract}
	
	\paragraph*{Objectifs} Dans un premier temps, l'objectif est de catégoriser les différents cafés en tentant de trouver des tendances gustatives par rapport aux conditions de culture. Dans un second temps, il faudra pouvoir prédire la qualité en bouche des cafés par rapport aux conditions environnementales. 
	
	\paragraph*{Tâches}
	\begin{itemize}
		\item Analyse du problème et planification des étapes du projet
		\item Analyse du contexte technique et scientifique ainsi que de l'état de l'art
		\item Décomposition du problème et conception de la solution
		\begin{enumerate}
			\item Prise en main et analyse des données disponibles
			\item Analyse et sélection des méthodes de modélisation
			\begin{enumerate}
				\item Méthodes pour la caractérisation
				\item Méthodes pour la prédiction
			\end{enumerate}
		\end{enumerate}
	\item Réalisation, implémentation et tests
	\begin{enumerate}
		\item Implémentation des méthodes pour la caractérisation (par ex. Clustering, PCA, SOM, ...)
		\item Implémentation des méthodes pour la prédiction (par ex. Réseaux de neurones, Random Forest, Logique floue,...)
	\end{enumerate}

	\item Analyse des informations obtenues et discussion des résultats
	
	\item Document et présentation
	\end{itemize}
	
\end{abstract}











