\chapter{Analyse des données}


\section{Data Mining - Analyse exploratoire}

\subsection{Pre-processing}
%Description du dataset (rapidement, le reste dans un codebook)\\

%Mise en forme du dataset, une sortie à la fois vs plusieurs sorties etc \\
%On considère les données gustatives et les défauts physiques comme des sorties -> les prendre un par un ou tous ensemble\\

%Description du set, différentes parties des données, éventuelles données manquantes\\

Une grande partie de l'étape de préprocessing a été réalisée durant l'extraction des données. Il a en effet fallu extraire les données d'une manière uniforme et cohérente dès le départ afin de ne pas se retrouver avec des variables présentes uniquement dans certaines parties du set de données ou avec des variables incohérentes comme cela a été le cas de certains cafés qui avaient comme note de dégustation plus de cent points sur cent par exemple. Des éliminations ou des corrections ont été réalisées de manière automatisées ou manuelles afin de supprimer les erreurs. On notera parmi les corrections importantes le remplacement de virgules par des points, quelques erreurs de frappes (68.5 points sur 10 au lieu de 6.85 par exemple) ou encore des utilisations d'unités différentes. Le dataset résultant est décrit dans l'Annexe A de ce projet. Une fois cette étape de nettoyage réalisée, les premières informations ont pu être extraites des données.


\noindent Le schéma \ref{DatasetMaking} résume rapidement les différentes élimination d'observations au cours des étapes de construction du set de données. 

\begin{figure}[H]
	\includegraphics[scale=0.5]{Dataset_1}
	\caption{\label{DatasetMaking} Étape de construction du set de données et pertes d'observations.}
\end{figure}


\paragraph{Variables dépendantes\label{VarDep}} Les variables de sorties, ou dépendantes, sont composées des dix analyses de dégustation et éventuellement de la totalité des défauts physiques des grains. Sauf pour les défauts physiques des grains, qui sont éliminés avant la dégustation, les variables de sorties sont très corrélées entre elles et il a été discuté avec le responsable du comité  de sélectionner les variables les plus importantes. Ces variables sont \textit{Acidez}, \textit{Dulzor} et \textit{Puntaje Total} ainsi que la catégorie décrite dans le paragraphe suivant qui a été ajouté au set de données.  

\paragraph{Catégories}
Les cafés colombiens sont réputés excellents mais sur les 100 points attribuable lors des dégustations qu'est-ce que cela représente ? Le tableau \ref{categoriesCafe} nous donne une bonne idée de l'échelle représentée. Ces catégories seront utilisées dans le set de données pour tenter de réaliser une classification. 
\begin{table}[H]
	\centering
	\caption{Catégories de cafés d'après le nombre de points \newline Source: http://www.scaa.org/?page=resources\&d=cupping-protocols \label{categoriesCafe}}
	\begin{tabular}{llllcl}
		\textbf{Total Score}   &  & \textbf{Quality Classification}   & \textbf{Specialty or not}  & \textbf{Category} \\
		
		90-100        &  & Outstanding              & Specialty         &  1                    \\
		85-89.99      &  & Excellent                & Specialty         &  2                    \\
		80-84.99      &  & Very Good                & Specialty         &  3                    \\
		\textless80.0 &  & Below Specialty Quality & Not Specialty     &  4                   
	\end{tabular}
\end{table}









\subsection{Analyse exploratoire}


%TODO Cases des observations - peuvent être recyclées après
\subsubsection{Emplacement des fermes}\label{EmpFermes} L'emplacement des différents cafés par rapport aux nombre de points est présenté sur la figure \ref{FincaVSPoints}. La catégorie 1 (non représentée) correspond aux cafés avec plus de 90 points, la 2 aux cafés avec plus de 85, la 3 aux cafés avec plus de 80 et la 4 aux cafés en dessous de 80, ne correspondant donc pas à la qualité "Specialty". On peut y voir que l'emplacement dans le département n'as pas d'incidence sur les résultats, la répartition des différentes classes étant très uniforme. 


\begin{figure}[H]
	\includegraphics[scale=0.59]{Map_North}
	\newline
	\newline
	\newline
	\includegraphics[scale=0.5]{Map_South}
	
	\begin{figure}[H]
		\includegraphics[scale=1]{Legend_Map2}
	\end{figure}
	\caption{\label{FincaVSPoints} Emplacement des fermes avec coloration selon le nombre de points attribués au café.}
\end{figure}


\subsubsection{Corrélations entre variables}

Afin d'avoir une bonne vue d'ensemble sur les variables et leurs liens, les matrices de corrélation ont été calculées pour toutes les variables. Premièrement la corrélation entre les différentes sorties. Sur la figure \ref{correlation_sorties1} on peut observer que les défauts physiques des grains ne sont que très peu liés entre eux ou avec les résultats de dégustation. On remarque cependant que les données gustatives du café sont fortement liées entre elles. 

\begin{figure}[H]
	\includegraphics[scale=0.35]{correlation_sorties1}
	\caption{\label{correlation_sorties1} Matrice de corrélation entre les différentes sorties.}
\end{figure}


\noindent Sur la figure \ref{correlation_all1} on peut observer les corrélations entre toutes les variables. 


\begin{figure}[H]
	\includegraphics[scale=0.35]{correlation_all1}
	\caption{\label{correlation_all1} Matrice de corrélation entre toutes les variables.}
\end{figure}


\noindent On observe que les données de sol sont parfois corrélées entre-elles mais presque pas du tout avec les données climatiques. On peut donc déjà soulever que le climat n'a pas ou peu d'influence sur la texture, le pH ou le taux de matière organique du sol des zones étudiées. Les précipitations ont une influence importante sur les défauts physiques des grains, en revanche nous n'avons aucune variable qui a une corrélation marquée avec les points totaux et donc la qualité du café.   



%TODO Parler des variétés de cafés et de quelques données les concernants (Moyennes de points, rendement par variété, altitude etc) 









%TODO vérifier la version -> ne pas inclure les données de sorties !
\newpage
\subsubsection{Principal Component Analysis (PCA) et Clustering}\label{PCAss}
La PCA, pour Analyse en Composantes Principales en français, est une méthode qui consiste à transformer un jeu de variables corrélées en nouvelles variables dé-corrélées les unes des autres. Ces nouvelles variables sont appelées composantes principales et permettent de rendre l'information moins redondante. Pour faire plus simple, l'utilité de la Composante Principale est de réduire le nombre de variables tout en gardant un maximum d'information. La figure \ref{PCAdefinition} montre une représentation graphique de la composante principale. 


\begin{figure}[H]
	\includegraphics[scale=0.5]{PCA_1}
	\caption{\label{PCAdefinition} Description de l'Analyse en Composante Principale. (A) Description d'un objet simple de manière compliquée ( trois dimensions pour par exemple une ellipse en papier) (B) Trouver des nouvelles variables (axes de coordonnées) orthogonaux l'un à l'autre qui pointent dans les directions de la plus grande variance (C) Utiliser les nouvelles variables (axes) pour décrire l'objet d'une manière plus simple. }
\end{figure}

\paragraph{Résultats de la PCA}

L'analyse sur une version compacte des variables a donné les résultats présentés dans le tableau \ref{TablePCAResult1} et sur la figure \ref{FigurePCAResult1}. Une première analyse avait été effectuée sur le set de données complet, c'est-à-dire avec la totalité des données climatiques et non les moyennes, et les résultats se sont avérés similaires mais plus difficilement lisibles. Il a donc été choisi de résumer les variables pour réaliser la PCA et le clustering.   

\begin{table}[]
	\centering
	\begin{tabular}{lllllll}
		& PC1     & PC2     & PC3     & PC4     & PC5     & PC6     \\
		ASNM           & \textbf{0.4125}  & 0.0766  & -0.0360 & 0.1766  & -0.1337 & 0.0397  \\
		Luminosidad    & -0.0169 & 0.2320  & 0.1532  & -0.3178 & -0.2836 & 0.0157  \\
		PrecTotalAvg   & -0.1714 & 0.1424  & 0.0222  & \textbf{-0.6767} & -0.0667 & 0.0926  \\
		TminTotalAvg   & \textbf{-0.4589} & -0.0115 & 0.1086  & -0.1232 & 0.0013  & 0.0809  \\
		TmaxTotalAvg   & \textbf{-0.4745} & -0.0334 & 0.0296  & 0.1501  & -0.0253 & -0.0329 \\
		TmeanTotalAvg  & \textbf{-0.4803} & -0.0252 & 0.0631  & 0.0409  & -0.0149 & 0.0133  \\
		DtrTotalAvg    & -0.3323 & -0.0517 & -0.0884 & \textbf{0.4712}  & -0.0529 & -0.1769 \\
		OrientationNum & 0.0623  & 0.0592  & -0.0886 & -0.1839 & \textbf{0.3797}  & \textbf{-0.5662} \\
		Slope          & 0.0657  & -0.1842 & 0.1179  & 0.1047  & 0.0434  & \textbf{0.6905}  \\
		pH\_avg        & 0.0135  & 0.3892  & 0.3844  & -0.0168 & 0.1244  & 0.0679  \\
		org\_avg       & 0.0493  & 0.1781  & \textbf{0.5036}  & 0.2176  & -0.1639 & -0.1841 \\
		Franco         & -0.0157 & \textbf{0.5453}  & 0.1441  & 0.1921  & 0.1676  & 0.1108  \\
		Arcilloso      & -0.0180 & -0.3026 & 0.2993  & -0.0844 & \textbf{0.4532}  & 0.1751  \\
		Limoso         & -0.0754 & \textbf{0.5018}  & \textbf{-0.2016} & 0.1095  & 0.2726  & 0.1712  \\
		Arenoso        & -0.0210 & 0.0645  & 0.0180  & 0.0048  & \textbf{-0.6303} & -0.0156 \\
		Cascajoso      & 0.0760  & -0.2236 & \textbf{0.6130}  & 0.0095  & 0.0109  & \textbf{-0.2053}
	\end{tabular}
	\caption{\label{TablePCAResult1}Tableau des rotations des six premiers composants de la PCA avec mise en évidence des variables les plus importantes par composante.}
\end{table}


\begin{figure}[H]
	\includegraphics[scale=1]{pca_simplified.pdf}
	\caption{\label{FigurePCAResult2} Résultats de la PCA sous forme graphique. Réalisé avec une partie simplifiée des variables }
\end{figure}

\begin{figure}[H]
	\includegraphics[scale=1]{pca_complete.pdf}
	\caption{\label{FigurePCAResult1} Résultats de la PCA sous forme graphique. Réalisé avec la totalité des variables }
\end{figure}


\paragraph{Analyse des composantes} Le tableaux \ref{TablePCAResult1} montre l'importance des variables dans les différentes composantes de la PCA réalisée avec un  jeu de variables simplifié pour plus de lisibilité, après vérification que l'interprétabilité des deux tableaux était similaire. \\

\noindent La première composante met en évidence les températures par rapport à l'altitude alors que la deuxième et la troisième mettent en évidence principalement les caractéristiques du sol. La quatrième montre une dé-corrélation entre les précipitations moyennes et les DTR et la cinquième composante montre une corrélation entre la texture argileuse du sol et son orientation à l'opposé à d'un sol sablonneux. La sixième composante met en évidence une relation entre l'orientation et et les sols rocailleux à l'opposé des sols pentus. 



% https://georgemdallas.wordpress.com/2013/10/30/principal-component-analysis-4-dummies-eigenvectors-eigenvalues-and-dimension-reduction/

%https://onlinecourses.science.psu.edu/stat505/node/54
\paragraph{Clustering} Afin de vérifier la présence de groupes d'individus parmi la population de café, nous réalisons un HCPC pour \textit{Hierarchical Clustering on Principal Components} à l'aide de la PCA. 

\paragraph{Kruskal-Wallis}






\newpage
\subsubsection{Self-Organizing Map}
%TODO 

\paragraph{U-Matrix, répartition des cafés par classes et composants} 
La carte auto-organisatrice a été réalisée en incluant toutes les variables afin de voir où se placent les variables dépendantes par rapport aux variables indépendantes et de permettre d'observer d'éventuels clusters. 


\begin{figure}[H]
	\centering
	\includegraphics[width=0.7\linewidth]{SOM/SOM_Umatrix}
	\caption{U-Matrix de la carte SOM}
	\label{}
\end{figure}

\begin{figure}[H]
	\centering
	\includegraphics[width=1\linewidth]{SOM/SOM_Umatrix_Points_Category}
	\caption{U-Matrix avec les points. En jaune les cafés de catégorie 2,en bleu catégorie 3 et en rouge catégorie 4. Les catégories sont expliquées au point \ref{EmpFermes}}
	\label{}
\end{figure}


\begin{figure}[H]
	\caption{Composants de qualité - Variables de sortie}
	\centering
		\subfloat[Acidez]{\includegraphics[width=.3\linewidth]{SOM/SOM_Acidez}}\hfill
		\subfloat[Aroma-Fragrancia]{\includegraphics[width=.3\linewidth]{SOM/SOM_AromaFragrancia}}\hfill
		\subfloat[Cuerpo]{\includegraphics[width=.3\linewidth]{SOM/SOM_Cuerpo}}
		\newline
		\subfloat[Balance]{\includegraphics[width=.3\linewidth]{SOM/SOM_Balance}}\hfill
		\subfloat[Uniformidad]{\includegraphics[width=.3\linewidth]{SOM/SOM_Uniformidad}}\hfill
		\subfloat[Taza Limpia]{\includegraphics[width=.3\linewidth]{SOM/SOM_TazaLimpia}}
		\newline
		\subfloat[Dulzor]{\includegraphics[width=.3\linewidth]{SOM/SOM_Dulzor}}\hfill
		\subfloat[Sabor]{\includegraphics[width=.3\linewidth]{SOM/SOM_Sabor}}\hfill
		\subfloat[Sabor Residual]{\includegraphics[width=.3\linewidth]{SOM/SOM_SaborResidual}}
		\newline
		\centering
		\subfloat[Puntaje Catador]{\includegraphics[width=.3\linewidth]{SOM/SOM_PuntajeCatador}}\hfill
		\subfloat[Puntaje Total]{\includegraphics[width=.3\linewidth]{SOM/SOM_PuntajeTotal}}
\end{figure}




\begin{figure}[H]
	\caption{Précipitations}
	\centering
	\subfloat[Prec1]{\includegraphics[width=.3\linewidth]{SOM/SOM_Prec1}}\hfill
	\subfloat[Prec2]{\includegraphics[width=.3\linewidth]{SOM/SOM_Prec2}}\hfill
	\subfloat[Prec3]{\includegraphics[width=.3\linewidth]{SOM/SOM_Prec3}}\hfill
	\newline
	\subfloat[Prec4]{\includegraphics[width=.3\linewidth]{SOM/SOM_Prec4}}\hfill
	\subfloat[Prec5]{\includegraphics[width=.3\linewidth]{SOM/SOM_Prec5}}\hfill
	\subfloat[Prec6]{\includegraphics[width=.3\linewidth]{SOM/SOM_Prec6}}\hfill
	\newline
	\subfloat[Prec7]{\includegraphics[width=.3\linewidth]{SOM/SOM_Prec7}}\hfill
	\subfloat[Prec8]{\includegraphics[width=.3\linewidth]{SOM/SOM_Prec8}}\hfill
	\subfloat[Prec9]{\includegraphics[width=.3\linewidth]{SOM/SOM_Prec9}}\hfill
	\newline
	\centering
	\subfloat[Prec10]{\includegraphics[width=.3\linewidth]{SOM/SOM_Prec10}}
	\subfloat[Average]{\includegraphics[width=.3\linewidth]{SOM/SOM_PrecTotalAvg}}
\end{figure}



\begin{figure}[H]
	\caption{SOM - Autres données}
	\centering
	\subfloat[Temp. Min. Average]{\includegraphics[width=.3\linewidth]{SOM/SOM_TminTotalAvg}}\hfill
	\subfloat[Temp. Max. Average]{\includegraphics[width=.3\linewidth]{SOM/SOM_TmaxTotalAvg}}\hfill
	\subfloat[Temp. Mean. Average]{\includegraphics[width=.3\linewidth]{SOM/SOM_TmeanTotalAvg}}\hfill
	\newline
	\subfloat[Dtr Average]{\includegraphics[width=.3\linewidth]{SOM/SOM_DtrTotalAvg}}\hfill
	\subfloat[ASNM]{\includegraphics[width=.3\linewidth]{SOM/SOM_ASNM}}\hfill
	\subfloat[Slope]{\includegraphics[width=.3\linewidth]{SOM/SOM_Slope}}\hfill
	\newline
	\subfloat[Franco]{\includegraphics[width=.3\linewidth]{SOM/SOM_Franco}}\hfill
	\subfloat[Arenoso]{\includegraphics[width=.3\linewidth]{SOM/SOM_Arenoso}}\hfill
	\subfloat[Arcilloso]{\includegraphics[width=.3\linewidth]{SOM/SOM_Arcilloso}}\hfill
	\newline
	\centering
	\subfloat[Limoso]{\includegraphics[width=.3\linewidth]{SOM/SOM_Limoso}}\hfill
	\subfloat[Cascajoso]{\includegraphics[width=.3\linewidth]{SOM/SOM_Cascajoso}}\hfill
	\subfloat[pH]{\includegraphics[width=.3\linewidth]{SOM/SOM_pHAvg}}\hfill
	\newline
	\subfloat[Organic]{\includegraphics[width=.3\linewidth]{SOM/SOM_OrgAvg}}\hfill
	\subfloat[Orientation]{\includegraphics[width=.3\linewidth]{SOM/SOM_OrientationDegrees}}\hfill
	\subfloat[Defectos]{\includegraphics[width=.3\linewidth]{SOM/SOM_DefectosTotales}}
\end{figure}



\begin{figure}[H]
	\caption{\label{SecondSOMASNM}Altitude et défauts lors d'une seconde exécution du réseau de neurones}
	\centering
	\subfloat[ASNM]{\includegraphics[width=.5\linewidth]{SOM/SOM_ASNM2}}\hfill
	\subfloat[Defectos Totales]{\includegraphics[width=.5\linewidth]{SOM/SOM_DefectosTotales2}}
	\newline

	\centering
	\subfloat[ASNM]{\includegraphics[width=.5\linewidth]{SOM/SOM_ASNM2}}\hfill
	\subfloat[Puntaje Total]{\includegraphics[width=.5\linewidth]{SOM/SOM_PuntajeTotal}}\hfill
	\newline
	\centering 
	\subfloat[Coffee Class Position]{\includegraphics[width=.5\linewidth]{SOM/SOM_Umatrix_Points_Category}}
\end{figure}

\newpage
\paragraph{Analyse des cartes}On peut observer que les régions avec beaucoup de précipitations sont aussi celles qui ont le plus de cafés avec peu de points. Lors d'une seconde exécution du réseau on a pu observer une relation proche entre les défauts et l'altitude comme montré sur la figure \ref{SecondSOMASNM}. Cette relation n'était pas clairement visible lors de la première exécution. Le fait est déjà connu que le café qui pousse en altitude est de meilleure qualité cependant ici on remarque que les cafés cultivés aux altitudes les plus hautes sont les plus sujets à une grande quantité de défauts. Les défauts physiques n'interviennent que peu dans la qualité finale du café mais peuvent avoir une influence en cas de mauvais tri. On observe sur la figure \ref{AltPts} que la zone en bas à gauche, là où l'altitude est la plus haute, correspond à une zone possédant peu de cafés ayant eu le score 0 et beaucoup ayant eu des scores entre 85 et 90. Cependant, des cafés ayant un nombre de points de moins de 80 points, en rouge ou compris entre 89 et 85, en bleu, sont présents plus ou moins de manière homogène sur la totalité de la carte, indépendamment de l'altitude.  









\newpage

\section{Résultats des analyses  }
%Il a été possible ou non de faire de la prédiction, confusion matrix etc
%ce que les méthodes comme random forest ont donné
Le but de cette section est d'analyser la possibilité de prédiction de la qualité des cafés à l'aide des données climatiques et de qualité de sols à disposition. La qualité du café fait références aux variables dépendantes citées dans la section \ref{VarDep}.

\subsection{Perceptron Multi-couches}
%TODO sur R pour comparer avec les mêmes datasets
Pendiente - pending


\subsection{Random Forest}
% analyser des variables séparément. Possibilité de prédir les défauts vs le "gout", les défauts spéarément - les gouts séparément etc 






\subsection{Partial Least Squares}
% analyser les résultats de prédiction ainsi que les corrélation fournies par PLS - au vus des résultats et de la corrélation entre les variables de sorties pas besoind 'utiliser MBPLS ça serait une perte de temps



\chapter{Discussion des résultats}


%Analyse globale des résultats, discussion sur les résultats 
%Données manquantes

%Exemple: on sait que la fermentation du grain dans sa pulpe impacte sur la chimie de la graine et peut rendre le café plus ou moins acide -> pas de données sur la fermentation (temps, méthode etc) idem pour le séchage, le stockage, la récolte etc
