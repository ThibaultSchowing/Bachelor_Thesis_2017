\chapter{introduction}
\section{Question de recherche}

A partir de données sur le climat, la qualité du sol et les pratiques culturales, est-il possible d’expliquer et de prédire les différents traits de la qualité en bouche des cafés du département de Risaralda ?



\section{Contexte du projet}

Le sujet de ce Travail de Bachelor a été proposé par le « Centro Internacional de Agricultura Tropical » (CIAT) qui travaille dans le but d’améliorer la productivité et la gestion de l’agriculture en zone tropicale, et dont les bureaux se trouvent à Cali, en Colombie.

À 200 kilomètres de Cali, le comité des caféiculteurs de Risaralda souhaite pouvoir expliquer les différents traits de la qualité en bouche des cafés produits dans les différents secteurs de leur département. La filière café colombienne est en effet en concurrence avec d’autres pays exportateurs sur le marché international, et un des avantages comparatifs de la Colombie est que ses terroirs produisent des cafés de qualité et de caractères affirmés. Il est donc stratégique pour la fédération des caféiculteurs de Colombie d'être en mesure de faire valoir ces spécificités pour aller chercher la valeur ajoutée associée aux produits démarqués du lot.

Ce projet a pour but de trouver des méthodes de modélisation afin d’identifier les caractéristiques du café spécifiques à chaque secteur de la région en se basant sur des analyses gustatives, des données climatiques et géographiques, et d’autres données de pratiques culturales.

Dans un premier temps, l’objectif est de catégoriser les cafés en tentant de trouver des tendances gustatives par rapport aux conditions de culture. Dans un second temps, il faudra pouvoir prédire la qualité en bouche des cafés par rapport aux conditions environnementales.

Le but de cette collaboration sur le long terme est de permettre au département de Risaralda de mettre en valeur la diversité de ses cafés, principalement à des fins de promotion auprès des acheteurs. 

\section{Contexte des données}


