\documentclass[11pt,a4paper,twoside,openright]{report}

% Packages 

\usepackage[francais]{babel}
\usepackage[utf8]{inputenc}

\usepackage{amsmath}
\usepackage{amsfonts}
\usepackage{amssymb}
\usepackage{makeidx}
\usepackage{graphicx}
\usepackage{afterpage}
\usepackage{cite}
\usepackage{longtable}

\usepackage[section]{placeins}
\usepackage{float}
\usepackage{listings}
\usepackage{color}

\usepackage{booktabs} % To thicken table lines
\usepackage{pgfplotstable}
\usepackage[final]{pdfpages}


\usepackage[hidelinks]{hyperref}
\usepackage{minitoc}

\usepackage{cleveref}

\usepackage{geometry} 



\lstset{frame=tb,
	language=R,
	keywordstyle=\color{blue},
	alsoletter={.}
}

\usepackage{enumitem}
\renewcommand\descriptionlabel[1]{\textbf{#1 :}}

\usepackage{subfig}
\usepackage{graphicx}

\usepackage{array}

% Evite les gros début de chapitre inutiles. 
\usepackage{titlesec}

% niveaux de table des matières
\setcounter{secnumdepth}{3}
\setcounter{tocdepth}{3}

% Liste des abréviations
\usepackage{nomencl} 
\makenomenclature 
\renewcommand{\nomname}{Liste des abréviations}% Pour redéfinir le titre De cette liste





\makeatletter

\newif\if@mainmatter \@mainmattertrue

\newcommand\frontmatter{%
	\cleardoublepage
	\@mainmatterfalse
	\pagenumbering{roman}}
\newcommand\mainmatter{%
	\cleardoublepage
	\@mainmattertrue
	\pagenumbering{arabic}}
\newcommand\backmatter{%
	\if@openright
	\cleardoublepage
	\else
	\clearpage
	\fi
	% \@mainmatterfalse
}
\makeatother


\titleformat{\chapter}
{\normalfont\LARGE\bfseries}{\thechapter}{1em}{}
\titlespacing*{\chapter}{0pt}{3.5ex plus 1ex minus .2ex}{2.3ex plus .2ex}

% Modification de commandes 

\newcommand\blankpage{%
	\null
	\thispagestyle{empty}%
	\addtocounter{page}{-1}%
	\newpage}

\graphicspath{ {img/} }


\author{Thibault Schowing}

%TODO Trouver un bon titre au travail
%\title{Amélioration de la productivité de cultures tropicales par des méthodes d'apprentissage automatique}
%\title{Analyse des relations entre les conditions de culture et la qualité des cafés du département de Risaralda en Colombie }


% Mise en page et marges
% https://tex.stackexchange.com/questions/42063/illogical-twoside-margins


\begin{document}
	% Ordre de préférence:
	% a) La page de couverture
	% b) Le cahier des charges
	% c) La table des matières
	% d) Le résumé
	% e) L'introduction
	% f) Le corps du rapport
	% g) La conclusion
	% h) La bibliographie
	% -i) La liste des symboles et abréviations utilisés
	% -j) La liste des figures
	% -k) Les annexes
	% -l) Le journal de travail
	
	% Page de titre
	
	\dominitoc
	\newgeometry{hmarginratio=1:1}
	\begin{titlepage}
		\centering
		
		\small{Haute Ecole d'Ingénierie et de Gestion du Canton de Vaud  \par}
		\footnotesize{University of Applied Sciences Western Switzerland\par}
		\vspace{1cm}
		
		\includegraphics[width=0.5\textwidth]{HEIG-VDLogo}\par
		
		\vspace{1cm}
		\Large{Analyse des relations entre les conditions de culture et la qualité des cafés du département de Risaralda en Colombie\par}
		\vspace{1.5cm}
		\small{Caractériser et prédire la qualité des cafés colombiens \par}
		\vspace{2cm}
		\small\textit{Thibault \textsc{Schowing}}\par
		\small{Travail de Bachelor}\par
		\small{\today\par}
		
		\vfill
		Professeur responsable: ~Carlos Andrès \textsc{Peña}\par
		Superviseur (CIAT): Sylvain \textsc{Delerce} \par
		Superviseur (CIAT): Daniel \textsc{Jimenez}
		
		
	\end{titlepage}
	\restoregeometry 
	%\afterpage{\blankpage}
	\pagenumbering{gobble}
	
	\frontmatter
	
	% Entrées pour la liste des abréviations, compiler -> exécuter la commande ci-dessous -> recompiler
	
	% makeindex -s nomencl.ist -t "Bachelor_Schowing2017.nlg" -o "Bachelor_Schowing2017.nls" "Bachelor_Schowing2017.nlo"
	\nomenclature{SOM}{Self Organizing Map}
	\nomenclature{FNC}{Fédération Nationale des Caféiculteurs}
	\nomenclature{DTR}{Diurnal Temperature Range}
	\nomenclature{SICA}{Sistema de Información Cafetera}
	\nomenclature{SCAA}{Specialty Coffee Association of America}
	\nomenclature{CIAT}{Centre Internationnal de recherche pour l'Agriculture Tropicale}
	\nomenclature{tmax}{Température maximale}
	\nomenclature{tmin}{Température minimale}
	\nomenclature{tmean}{Température moyenne}
	\nomenclature{GIS}{Geographical Information System}
	\nomenclature{HCPC}{Hierarchical Clustering on Principal Components}
	\nomenclature{}{}
	
	
	\afterpage{\blankpage}
	\renewcommand{\abstractname}{Remerciements}
\begin{abstract}
	\thispagestyle{plain}
	\noindent Je tiens à adresser mes remerciements à tous ceux qui m'ont accompagné dans la réalisation de ce projet, sans eux, rien n'aurait été possible. \\
	
	\noindent Un grand merci au professeur Carlos Andrés Peña, qui m'a donné l'opportunité de sortir des sentiers battus et de découvrir un environnement de travail exceptionnel à Cali en Colombie. C'est une expérience que je ne suis pas près d'oublier et qui va sans doute longtemps me suivre dans mon parcours professionnel et personnel.\\
	
	\noindent Merci à Sylvain, Hugo et Daniel qui m'ont guidé à travers ce projet grâce à leur grande expérience dans le domaine. Grâce à vous j'ai appris énormément et ce bagage me sera d'une grande utilité dans le futur.\\

	\noindent Merci à la famille et aux amis pour leur temps de relecture et leur patience.\\
	
	\noindent Et enfin, un grand merci à Fanny, Alexandra, Hugo, Andrés, Steven et tous les autres qui m'ont immédiatement accepté et avec qui j'ai vécu une grande aventure de deux mois, du Pacifique aux montagnes de l'axe du café. 
	
	% Une petite citation ça serait cool, mais à part "garbage in garbage out" je vois pas.
	
\end{abstract}
	\afterpage{\blankpage}
	
	\renewcommand{\abstractname}{Cahier des Charges}
\begin{abstract}
	
	\paragraph*{Objectifs} Dans un premier temps, l'objectif est de catégoriser les différents cafés en tentant de trouver des tendances gustatives par rapport aux conditions de culture. Dans un second temps, il faudra pouvoir prédire la qualité en bouche des cafés par rapport aux conditions environnementales. 
	
	\paragraph*{Tâches}
	\begin{itemize}
		\item Analyse du problème et planification des étapes du projet
		\item Analyse du contexte technique et scientifique ainsi que de l'état de l'art
		\item Décomposition du problème et conception de la solution
		\begin{enumerate}
			\item Prise en main et analyse des données disponibles
			\item Analyse et sélection des méthodes de modélisation
			\begin{enumerate}
				\item Méthodes pour la caractérisation
				\item Méthodes pour la prédiction
			\end{enumerate}
		\end{enumerate}
	\item Réalisation, implémentation et tests
	\begin{enumerate}
		\item Implémentation des méthodes pour la caractérisation (par ex. Clustering, PA, SOM, ...)
		\item Implémentation des méthodes pour la prédiction (par ex. Réseaux de neurones, Random Forest, Logique floue,...)
	\end{enumerate}

	\item Analyse des informations obtenues et discussion des résultats
	
	\item Document et présentation
	\end{itemize}
	
\end{abstract}












	\afterpage{\blankpage}
	
	
	%\section*{Résumé}
%TODO résumé
%Résumé
\renewcommand{\abstractname}{Résumé}
\begin{abstract}
	This is cool paper about vuvuzelas.
\end{abstract}



	
	
	\tableofcontents
	
	

	
	\mainmatter
	
\chapter{Introduction}
\section{Question de recherche}

A partir de données sur le climat, la qualité du sol et les pratiques culturales, est-il possible d’expliquer et de prédire les différents traits de la qualité en bouche des cafés du département de Risaralda ?


\section{Contexte du projet}
Le sujet de ce Travail de Bachelor a été proposé par le « \textit{Centro Internacional de Agricultura Tropical }» (CIAT) qui travaille dans le but d’améliorer la productivité et la gestion de l’agriculture en zone tropicale, et dont les bureaux se trouvent à Cali, en Colombie.\\

\noindent À 200 kilomètres de Cali, le comité des caféiculteurs de Risaralda souhaite pouvoir expliquer les différents traits de la qualité en bouche des cafés produits dans les différents secteurs de leur département. La filière café colombienne est en effet en concurrence avec d’autres pays exportateurs sur le marché international, et un des avantages comparatifs de la Colombie est que ses terroirs produisent des cafés de qualité et de caractères affirmés. Il est donc stratégique pour la fédération des caféiculteurs de Colombie d'être en mesure de faire valoir ces spécificités pour aller chercher la valeur ajoutée associée aux produits démarqués du lot.\\


\noindent Ce projet a pour but de trouver des méthodes de modélisation afin d’identifier les caractéristiques du café spécifiques à chaque secteur de la région en se basant sur des analyses gustatives, des données climatiques et géographiques, et d’autres données de pratiques culturales.\\


\noindent Dans un premier temps, l’objectif est de catégoriser les cafés en tentant de trouver des tendances gustatives par rapport aux conditions de culture. Dans un second temps, il faudra pouvoir prédire la qualité en bouche des cafés par rapport aux conditions environnementales.\\


\noindent Le but de cette collaboration sur le long terme est de permettre au département de Risaralda de mettre en valeur la diversité de ses cafés, principalement à des fins de promotion auprès des acheteurs. \\


\chapter{À propos des données}

\section{Extraction, description et contextualisation des données}

\subsection{Le système SICA}Le système SICA, pour \textit{Sistema de Información Cafetera}, est un système géré par la Fédération Nationale des Caféiculteurs (FNC), permettant d'identifier chaque parcelle de production de café en Colombie. C'est un système d'information d'envergure national, accessible via internet permettant de mettre à jour, consulter, analyser, modéliser et visualiser les données géospatiales sur les producteurs et les fermes de beaucoup de caféiculteurs du pays. C'est l'outil d'information stratégique pour la conception, le développement, la cartographie et le suivi des politiques de compétitivité et de la durabilité du café colombien\cite{SICA}. Chaque ferme possède un identifiant SICA, qui sera utilisé dans ce travail comme identifiant unique pour définir un café. Il est important car c'est ce numéro qui permet, via les services de la FNC, d'avoir un identifiant unique pour chaque parcelle et d'y associer des informations la concernant.  

\subsection{Données de pratiques culturales}
Malheureusement, aucune données de ce type n'a été fournies pour la réalisation de cette analyse. Les seules données à disposition sont des données de pratiques culturales générales issue de la littérature ou d'expériences personnelles.

\subsection{Données gustatives}
Les données gustatives sont très relatives aux sens et à la perception de chaque goutteur. Cependant, la SCAA, \textit{Speciality Coffee Association of America}, dispose d’un système de notation basé sur des hypothèses communautaires reconnues ce qui permet d’avoir une certaine régularité dans les données de dégustations. Les cafés sont notés sur 100 points répartis sur plusieurs critères: parfum/arôme, saveur, arrière-goût, acidité, corps, équilibre, douceur, clean-cup (absence de défauts marqués), uniformité et évaluation personnelle du testeur.  Chacun de ces critères est noté sur 10 mais aussi par des termes qualitatifs. Par exemple, la saveur, c’est-à-dire la combinaison de l’odeur et du goût, la première impression qu’on a en goûtant le café, peut être notée 7/10 et “Caramel”. \\

\noindent Un premier échantillon de trois cafés contenait toutes ces informations de manière uniforme, mais il s'est avéré que la partie mandante n'avait pas pu uniformiser la totalité des données brutes dans les délais et la tâche d'uniformisation nous a donc été confiée. Ainsi, les données finalement reçues variaient beaucoup d'un document à l'autre, d'une part dans les données de dégustations présentes et dans le type de document mais aussi dans les méta-données permettant d'identifier précisément de quel café il s'agissait. Il a donc fallut effectuer un tri et ne garder que la masse qu'il était possible d'utiliser. Les critères permettant de garder une dégustation ou non sont les suivants: Identification possible du café grâce au numéro SICA ou au numéro d'identité du caféiculteur, présence des défauts physiques du café, présence des caractéristiques gustatives de manière uniforme. La FNC a été sollicitée afin de compléter les données une fois celles-ci triées afin d'y ajouter les numéros SICA ou les numéros d'identité manquants, et d'y ajouter les coordonnées de chaque parcelle sous la forme de référence spatiale EPSG:3116 en suite converties en coordonnées GPS classiques degrés-décimaux.\\

\paragraph{Traitement du café} Pour avoir une vision d'ensemble, voici un petit résumé concernant la production du café dans une des fermes du département de Risaralda. Cette ferme ne reflète pas la production de toutes les fermes du département cependant elle fait partie des meilleures plantation du secteur, et c'était une occasion pour l'équipe de s'informer directement sur le terrain. \\

\noindent Lorsque les grains de café sont mûrs, ils sont récoltés à la main puis amené dans une grande cuve sous laquelle se trouvent les différentes machines permettant de traiter la baie afin d'en extraire le grain. La première de ces machine c'est la dépulpeuse qui permet d'enlever la partie charnue du grain. La pulpe est récupérée en contre-bas et le grain continue son chemin dans deux directions possibles. Si la ferme en est équipée, une machine appelée \textit{desmucilaginador} va enlever la matière gluante entourant le grain, appelé \textit{miel} ou en français \textit{mucilage}, en le lavant. Si la ferme n'est pas équipé de cette machine, les grains vont être déversés dans une cuve où un processus de fermentation va être lancé variant entre une dizaine d'heures à plusieurs jours ce qui aura pour effet de laver le mucilage des grains. Une fois les grains lavés, ils seront séchés soit à l'air en utilisant la chaleur du soleil dans des grandes terrasse à café, ce processus prends environ dix jours, soit dans des machines à air chaud, plus onéreuse mais permettant de sécher de grandes quantités de grains en quelques heures. Une fois les grains séchés, ils sont vendus et l'étape suivant consiste à retirer de manière industrielle les grains endommagés car un seul grain peut rendre une tasse imbuvable. Des machines analyse les grains et éliminent ceux dont la densité ou la couleur n'est pas normale.\cite{GuideCafe} \\

\noindent Les différentes méthodes de préparation du café ont chacune leurs avantages économiques, écologiques ou gustatifs. Par exemple, la taille de l'arbre, obligatoire après un certain nombre d'année, peut se faire de plusieurs manière chacune affectant le rendement de manière différente. La complexité chimique de la fermentation peut apporter certains arômes tout comme un séchage rapide à l'air chaud peut en enlever. %TODO citer qqch sur le café

 
\subsection{Données climatiques}
Les données climatiques comprennent les températures maximales, minimales et moyennes, la variation de température pendant la journée (DTR) et les quantités de précipitations. Les moyennes de ces mesures ont été calculée pour chaque mois et extrapolées sur une grande partie du territoire à partir de données provenant de différentes stations météorologiques du pays, permettant ainsi d’accéder aux mesures selon l’emplacement désiré à environ 500 mètres près. \\

\noindent En prenant par exemple les données de température maximale pour le mois de janvier 2011, en affectant pour les valeurs une échelle de couleurs, nous pouvons visualiser les données sous la forme d'une image comme sur la figure \ref{tmax_picture}.\\

\noindent Les données climatiques sont données de 2011 à 2016. il faudra cependant faire attention au fait qu'un café dégusté en février 2011 a poussé bien plus tôt. Les processus de récolte, de nettoyage, de fermentation, de séchage et de torréfaction du grain prennent du temps. Ce temps a dû être pris en compte afin de sélectionner les bonnes données et a été fixé à 10 mois sans prendre en compte le mois précédant la dégustation, qui est estimé comme temps nécessaire à la récolte et au traitement du grain. 


\begin{figure}[H]
	\centering
	\includegraphics[scale=1]{tmax_picture_1}
	\caption{\label{tmax_picture} Mise sous forme graphique du tableau des température maximales pour le mois de janvier 2011 }
\end{figure}

\paragraph{Contexte climatique Colombien}La Colombie se trouvant à proximité de l'équateur, on y trouve que deux saisons: l'été ou saison sèche (de décembre à janvier et de juillet à août) puis l'hiver ou saison des pluies (d'avril à mai et de octobre à novembre). Le relief du pays ainsi que sa taille, font varier le climat de chaud et humide pour la partie amazonienne et la région des caraïbes, désertique pour la région de Guajira tout au nord ou le désert de Tatacoa au centre et glacial pour les zones en haute altitude à plus de 3000 mètres. Le département de Risaralda se trouve dans le centre de la Colombie dans la région de l'Axe du café et jouit de conditions climatiques, géographiques et géologiques idéales pour la culture du café. Les températures oscillent entre 8 et 24 degrés mais le phénomène appelé \textit{El Niño} perturbe régulièrement le climat à l'échelle du continent, voir même du monde. 

% http://www.actulatino.com/2016/01/14/colombie-el-nino-provoque-chaleur-et-secheresse-juan-manuel-santos-appelle-a-la-vigilance/

% https://public.wmo.int/fr/medias/communiqu%C3%A9s-de-presse/d%E2%80%99une-intensit%C3%A9-exceptionnelle-l%E2%80%99%C3%A9pisode-el-ni%C3%B1o-amorc%C3%A9-son-d%C3%A9clin-mais


\paragraph{El Niño}El Niño désigne un phénomène climatique qui se caractérise par une augmentation des températures de l'eau dans l'est du Pacifique sud due à une perturbation dans la circulation atmosphérique entre les pôles et l'équateur. Ces perturbations déplacent les zones de précipitations, modifient les routes des cyclones ou typhons provoquent à certains endroits de fortes précipitations et à d'autres de longues périodes de sécheresse. Même dans les zones tempérées, les périodes El Niño changent les habitudes climatiques. Durant l'été austral 2015-2016 s'est produit un des épisodes El Niño le plus fort jamais enregistré\cite{OMM}. Si une grande partie de l'Amérique du Sud a été victime de fortes précipitations, la Colombie, elle, a subit une longue période de sécheresse et l'Europe a connu des records de chaleur. Sur la figure \ref{Nino} on peut observer les différents pics correspondants à l'intensité du phénomène ainsi que pour son opposé, La Niña. 

\begin{figure}[H]
	\centering
	\includegraphics[scale=0.25]{VariationElNino}
	\caption{\label{Nino} Intensité du phénomène El Niño au cours des ans\newline Source: \textit{http://ggweather.com/enso/oni.htm}}
\end{figure}

\paragraph{Impact du réchauffement climatique} Outre les phases d'El Niño, il est nécessaire de rappeler que le climat mondiale se réchauffe et que des conséquences se font ressentir. Le Centre du Commerce International \cite{CCI} nous donne un aperçu des conséquences que ce réchauffement pourrait avoir pour la Colombie. "Les coûts de production sont susceptibles d'augmenter en raison des nouvelles conditions climatiques favorisant la prolifération des insectes, invasions et microbes pathogènes, et perturbant à l'équilibre naturel entre certains parasites et leurs prédateurs naturels. Les maladies se développeront vers de nouvelles zones. Les besoins en eau peuvent augmenter en raison de températures plus élevées causant plus d'évaporation, forçant de nombreux agriculteurs à recourir à l'irrigation. Dans certaines régions, les agriculteurs voudront transférer leur production de café à de plus hautes altitudes afin de chercher d'un meilleur environnement."(Guide de l'Exportateur de Café, CCI, 2011 \cite{GuideCafe})


\subsection{Données de sols} 
Les données de qualité de sol sont subdivisées en profils. Chaque profil est séparé en une ou plusieurs couches d’une certaine profondeur dont sont renseignées les caractéristiques comme le pH, la texture ou encore le taux de matière organique. Les différentes textures sont présentées sur la figure \ref{TriangleTexture}. Afin d'avoir des données uniformes, les moyennes sur les 3 premières couches jusqu'à 1 mètre de profondeur ont été réalisées pour le pH et le niveau de matière organique alors que pour la texture, la somme des variable binaire a été effectuée. \\


\noindent Les données proviennent d'un GIS \footnote{Geographical Information System}, d'où il a été possible de croiser les données point par point afin d'extraire le profil de sol correspondant à un set de coordonnées GPS. Malheureusement les profils ne contiennent que pH, matière organique et texture. D'autres données très hétérogènes contenaient d'autres informations sur la composition chimique du sol mais leur structure et leur répartition irrégulière dans la zone de travail ont forcé à abandonner leur utilisation par manque de temps et de ressources.

% triangulo-del-suelo
\begin{figure}[H]
	\centering
	\includegraphics[scale=0.5]{triangulo-del-suelo}
	\caption{\label{TriangleTexture} Triangle représentant les différentes textures de sols\newline Source: \textit{http://www.construnatura.com/esp/articulo/agricultura-ecol-gica/el-suelo-como-fuente-de-vida--propiedades--ii-}}
\end{figure}




%################################################################################################################

\newpage
\section{Quelques chiffres et informations sur les données}

\paragraph{Emplacement des fermes}\label{EmpFermes} L'emplacement des différents cafés par rapport aux nombre de points est présenté sur la figure \ref{FincaVSPoints}. La catégorie 1 (non représentée) correspond aux cafés avec plus de 90 points, la 2 aux cafés avec plus de 85, la 3 aux cafés avec plus de 80 et la 4 aux cafés en dessous de 80, ne correspondant donc pas à la qualité "Specialty". On peut y voir que l'emplacement dans le département n'a, à première vue, pas d'incidence sur les résultats, la répartition des différentes classes étant très uniforme. Les différentes catégories attribuées aux cafés sont expliquées plus en détail au point \ref{categoriesCafe}.



\begin{figure}[H]
	\includegraphics[scale=0.59]{Map_North}
	\caption{Emplacement des fermes avec coloration selon le nombre de points attribués au café. Nord.}
\end{figure}

\begin{figure}[H]
	\includegraphics[scale=0.5]{Map_South}
	
	\begin{figure}[H]
		\includegraphics[scale=1]{Legend_Map2}
	\end{figure}
	\caption{\label{FincaVSPoints} Emplacement des fermes avec coloration selon le nombre de points attribués au café. Sud.}
\end{figure}


\newpage
\paragraph{Altitude et points} \footnote{Les sets de données utilisés ici peuvent différer, certaines données ont été extraites du set de base, contenant des données manquantes (pour certaines variables) et d'autres ont été extraites du set complet utilisé pour les calculs. Les données ne différent que très peu entre ces sets et l'impact est minimal pour le calcul de simple moyennes à titre exploratoire.}Les plantations sont réparties entre une altitude de 935 mètres et de 2001 mètres alors que nombre de points totaux attribués aux cafés varie entre 0 et 87.75. 

\begin{figure}[H]
	\centering
	\includegraphics[width=0.7\linewidth]{img/Exploration/plotAltitudeVsPuntajeTotal}
	\caption{Nombre de points totaux vs altitude}
	\label{fig:plotaltitudevspuntajetotal}
\end{figure}


\begin{figure}[H]
	\centering
	\includegraphics[width=0.7\linewidth]{img/Exploration/kdeplotPuntajeTotal}
	\caption{Répartition des cafés par rapport au nombre de points total}
	\label{fig:kdeplotpuntajetotal}
\end{figure}



\newpage
\paragraph{Précipitations et températures} Les deux années présentes dans les données, 2011 et 2016 sont deux années sensiblement différentes au niveau climatique, surtout au regard des précipitations.  

\begin{table}[H]
	\centering
	\caption{Précipitations par année}
	\label{YearlyPrec1}
	\begin{tabular}{lll}
		2011 & count                               & 350.000000 \\
		&mean & 192.985313                                      \\
		&std  & 28.732351                                       \\
		&min  & 108.813344                                     \\
		&25\% & 172.463001                                     \\
		&50\% & 199.349562                                      \\
		&75\% & 212.438373                                      \\
		&max  & 245.214644                                      \\
	2016 & count                               & 396.000000 \\
		&mean & 135.710853                                      \\
		&std  & 24.468758                                      \\
		&min  & 89.579520                                       \\
		&25\% & 119.952117                                      \\
		&50\% & 131.346726                                      \\
		&75\% & 140.572487                                      \\
		&max  & 189.482009                   
	\end{tabular}
\end{table}


\begin{table}[H]
	\centering
	\caption{Températures moyennes par année}
	\label{YearlyTmean}
	\begin{tabular}{lll}
		2011 & count                              & 350.000000 \\
		&mean & 20.969629                                      \\
		&std  & 1.195129                                       \\
		&min  & 18.106276                                      \\
		&25\% & 20.116997                                      \\
		&50\% & 20.880830                                      \\
		&75\% & 21.778835                                      \\
		&max  & 26.238706                                      \\
		2016 & count                              & 396.000000 \\
		&mean & 21.169157                                      \\
		&std  & 1.080422                                       \\
		&min  & 18.105124                                      \\
		&25\% & 20.549152                                      \\
		&50\% & 21.305588                                      \\
		&75\% & 21.918895                                      \\
		&max  & 24.874828  
	\end{tabular}
\end{table}

\newpage
\paragraph{Points totaux des dégustations} Des différences sont visibles par année en ce qui concerne les moyennes des points. En effet, l'année 2011 possède beaucoup plus de cafés notés avec zéro points que l'année 2016 par exemple.

\begin{table}[H]
	\centering
	\caption{Points totaux par année}
	\label{YearlyPuntajeTotal}
	\begin{tabular}{llll}
		2011 & count & 330.000000 &  \\
		&mean & 65.870833 &    \\
		&std & 30.274299 &    \\
		&min & 0.000000 &    \\
		&25\% & 76.000000 &    \\
		&50\% & 79.000000 &    \\
		&75\% & 82.000000 &    \\
		&max & 85.500000 &    \\
		2016 & count & 367.000000   \\
		&mean & 78.203597 &    \\
		&std & 11.202238 &    \\
		&min & 0.000000 &    \\
		&25\% & 77.375000 &    \\
		&50\% & 81.250000 &    \\
		&75\% & 83.750000 &    \\
		&max & 87.750000& 
	\end{tabular}
\end{table}




\begin{figure}[H]
	\centering
	\includegraphics[width=0.7\linewidth]{img/Exploration/PointTotauxEtc}
	\caption{Répartition des points totaux par année}
	\label{fig:pointtotauxetc}
\end{figure}



















%################################################################################################################

\chapter{Méthodes de modélisation}
\section{Rappel des objectifs}\label{obj}
Ce projet a deux principaux objectifs. Le premier est de trouver s'il existe différents groupes de cafés ayant des relations entre les conditions de culture et les caractéristiques physiques ou sensorielles. On cherche donc dans cette première partie à caractériser les cafés. On peut ici parler de clustering.  Le second objectif est de prédire les caractéristiques physiques ou sensorielles à partir des données sur les conditions de culture. Nous avons donc ici plusieurs possibilités de manières d'agir. Par exemple, si le clustering a réussi à diviser les cafés en différentes classes, on cherchera à prédire dans quelle classe se situe un nouveau café. Plus spécifiquement, on pourra se concentrer sur certains attributs du café, par exemple l'acidité, afin d'estimer quelle sera la note attribuée. 



\newpage

\section{Apprentissage supervisé}
% knn, réseaux de neurones etc
Le but de l'apprentissage supervisé est d'expliquer des sorties (outputs) à partir d'entrées (inputs). Des règles sont calculées à partir de données d'apprentissage selon différents modèles. Par la suite, le modèle est utilisé pour prédire des nouvelles données. On essayera ici d'expliquer les données gustatives du café ou ses défauts physiques à l'aide des données climatiques et de sols. 


\subsection{Random Forest}

La méthode Random Forest, ou \textit{forêts d'arbres décisionnels} en français, fait partie des méthodes ensemblistes\cite{EnsembleMethods}, qui utilisent la combinaison de plusieurs modèles de base, d'apprentissage automatique. Elle combine les concepts de sous-espaces aléatoires et de bagging.\\

\noindent Le bagging\label{bagging}, ou \textit{bootstrap agregation}, consiste à sous-échantillonner (ou ré-échantillonner au hasard avec doublons) le set d'entrainement et de faire générer à l’algorithme voulu un modèle pour chaque sous-échantillon. On utilise le bagging pour réduire la variance de la fonction de prédiction estimée. Le bagging semble bien fonctionner pour les procédures avec une grande variance et un petit biais, comme les arbres de décision. \cite{hastie_09_elements-of.statistical-learning}\\


\noindent Random Forest effectue donc un apprentissage sur de multiples arbres de décision entraînés sur des sous-ensembles de données légèrement différents \cite{Statistics01randomforests}.


\begin{figure}[H]
	\includegraphics[scale=0.7]{RandomForestSimple}
	\caption{\label{RandomForestSchema} Schéma simple du fonctionnement de Random Forest. \newline Source: \textit{https://www.youtube.com/watch?v=ajTc5y3OqSQ}}
\end{figure}

\subsection{Partial Least Square (PLS)} 

% https://www.youtube.com/watch?v=WKEGhyFx0Dg 
% https://www.utdallas.edu/~herve/abdi-wireCS-PLS2010.pdf 

PLS, originalement pour \textit{Partial Least Squares regression} puis plus récemment pour \textit{Projection to Latent Structures} est une méthode qui combine des propriétés de la PCA ainsi que de multiples régressions linéaires. Au lieu de trouver un hyperplan de la variance maximale, entre les variables dépendantes et indépendantes, cette méthode va trouver un modèle de régression linéaire en projetant les variables indépendantes et dépendantes dans un nouvel espace. Ce sont les variables latentes. Cette méthode est particulièrement utile lorsqu'il est nécessaire de prédire un jeu de variables dépendantes à partir d'un très grand jeu de variables indépendantes.  

\begin{figure}[H] 
	\centering
	\includegraphics[scale=0.5]{PLS_1} 
	\caption{\label{PLSschema}Méthode PLS. X est représenté par son score t et Y par u. Une première estimation de U est multipliée à travers X pour obtenir une aproximation du poid $ \omega_t $. Le poid est normalisé pour être de longueur 1 et remultiplié à travers X pour produire t. A partir de t et de Y, le poid $ q^T $ est obtenu ce qui donne un nouveau vecteur u. Cette opération est répétée jusqu'à la convergence de t.\cite{CEM:CEM515}} 
	% http://www.umb.no/statisk/specmod/mbseminar/Westerhuis1998.pdf 
\end{figure} 

\subsection{Multi Block PLS} 
% http://www.models.life.ku.dk/~courses/MBtoolbox/pres_IntroMultiBlock.pdf 

% https://books.google.ch/books?id=PPUbvBUvmWoC 

% http://www.umb.no/statisk/specmod/mbseminar/Westerhuis1998.pdf 

La PLS multi block est une extension de la méthode PLS qui sépare les variables indépendantes en plusieurs blocks afin de leur donner une plus grande interpértabilité et plus d'informations sur la structure générale des données. Dans le cadre de ce projet, on peut imaginer séparer les données climatiques des données de sol par exemple.   
L'exécution est très similaire à la méthode PLS classique.  

\begin{figure}[H] 
	\centering
	\includegraphics[scale=0.5]{MBPLS_1} 
	\caption{\label{MBPLSschema} Méthode MBPLS. Un score de départ u est régressé sur tous les blocs $ X_b $ pour donner les poids variables du bloc $ w^T_b $ Les poids des variables de blocs sont normalisés à la longueur un et multipliés par les blocs pour donner les scores de blocs $ t_b $.  Les scores de blocs sont combinés dans le super bloc T. Un cycle PLS entre T et Y est effectué pour donner le poids superieur $ W^T_T $, qui est également normalisé à la longueur un, et le super score $ t_T $. L'opération est répétée jusqu'à la convergence de $ t_T $. \cite{CEM:CEM515}} 
	% http://www.umb.no/statisk/specmod/mbseminar/Westerhuis1998.pdf 
\end{figure} 






\newpage

\section{Apprentissage non-supervisé}
Contrairement à l'apprentissage supervisé, l'apprentissage non-supervisé tente de trouver des groupes dans des données hétérogènes. Le but est d'extraire des connaissances à partir de ces données. Comme mentionné dans la partie \ref{obj}, notre but est de découvrir différents groupes de café identifiables. 


\subsection{SOM}


Les cartes SOM, pour \textit{Self Organizing Map} aussi appelées cartes de Kohonen du nom du statisticien ayant développé la méthode, sont des réseaux de neurones disposés en grille permettant de réduire dans un espace en deux dimensions des données ayant $n$ dimensions.\\
 

\noindent La carte est composée de composants appelés noeuds ou neurones. Ces neurones sont associés à un vecteur de poids de dimension identique au données d'entrée (initié aléatoirement) et à une position dans l'espace de la carte. L'algorithme va en suite trouver quelle neurone est le plus proche à la donnée pour chaque observation et va rapprocher les données du neurone dans l'espace de la carte et modifier les poids du neurone afin d'accroitre l'influence du neurone sur les prochaines observations proches\cite{Kohonen:1988:SFT:65669.104428}.\\


\noindent On souhaite vérifier s'il est possible de regrouper des cafés qui se distingueraient et à l'aide des différents composants de la carte SOM, comprendre quelles variables ont une influence sur la qualité. \\
 

\noindent Un bel exemple de SOM est celui de la carte de la pauvreté mondiale réalisé par le \textit{Department of Computer Science and Engineering} de l'université \textit{Helsinki University of Technology}. 

\begin{figure}[H]
	\includegraphics[scale=0.5]{SOMWordlPovertyMap}
	\caption{\label{SOMPovertyMap} Pays organisés en SOM d'après des indicateurs de pauvreté. \newline Source: \textit{http://www.cis.hut.fi/research/som-research/worldmap.html}}
\end{figure}

\begin{figure}[H]
	\includegraphics[scale=0.55]{worldmap}
	\caption{\label{WorldPovertyMap} Pays correspondants à la carte SOM de la figure \ref{SOMPovertyMap} \newline Source: \textit{http://www.cis.hut.fi/research/som-research/worldmap.html}}
\end{figure}



\newpage
\section{Optimisation}

%Boosting supprimé

\subsection{Cross-Validation}

Contrairement au bagging qui est utilisé pour réduire l'overfitting en entrainant plusieurs modèles sur des données ré échantillonnées (avec répétition) puis en construisant un modèle sur la moyenne de ces modèles, la cross-validation est utilisée pour tester la fiabilité d'un modèle en se basant sur un échantillonnage des données d'entrainement et de test. Il existe plusieurs méthodes: « holdout method », « k-fold cross-validation » et « leave-one-out cross-validation ».\\

\noindent La première consiste à diviser le set de données en deux et en utilisant une partie pour entrainer le modèle puis une autre pour le tester. L'erreur est estimée en calculant un score de performance avec une méthode comme MSE (Erreur Quadratique Moyenne ou \textit{Mean Square Error}). \\

\noindent Étant donné que les données sont souvent trop peu nombreuses pour se permettre de laisser tomber dès le départ une partie des données, la k-fold cross-validation devient utile. On divise le set en k échantillons puis on en sélectionne un comme étant le set de test puis les k-1 autres comme étant le set d'entrainement. On répète l'opération en sélectionnant chaque fois un échantillon différent pour le test. Le score de performance est calculé en réalisant la moyenne des scores des k validations effectuées. La méthode « Leave-one out » utilise le même principe mais en ne laissant qu'une seule entrée en dehors du set d'entrainement à chaque tour\cite{hastie_09_elements-of.statistical-learning}. 



	
	\chapter{Analyse des résultats}


\section{Data Mining - Exploration de données}


\subsection{Principal Component Analysis (PCA)}\label{PCAss}
La PCA, pour Analyse en Composantes Principales en français, est une méthode qui consiste à transformer un jeu de variables corrélées en nouvelles variables dé-corrélées les unes des autres. Ces nouvelles variables sont appelées composantes principales et permettent de rendre l'information moins redondante. Pour faire plus simple, l'utilité de la Composante Principale est de réduire le nombre de variables tout en gardant un maximum d'information. La figure \ref{PCAdefinition} montre une représentation graphique de la composante principale. 


\begin{figure}[H]
	\caption{\label{PCAdefinition} Description de l'Analyse en Composante Principale. (A) Description d'un objet simple de manière compliquée ( trois dimensions pour par exemple une ellipse en papier) (B) Trouver des nouvelles variables (axes de coordonnées) orthogonaux l'un à l'autre qui pointent dans les directions de la plus grande variance (C) Utiliser les nouvelles variables (axes) pour décrire l'objet d'une manière plus simple. }
	\includegraphics[scale=0.5]{PCA_1}
\end{figure}

On pourra ici éliminer les variables n'ayant pas une eigenvalue suffisamment importante afin de se concentrer sur les variables les plus importantes afin de réduire le nombre de dimension du dataset. 



% https://georgemdallas.wordpress.com/2013/10/30/principal-component-analysis-4-dummies-eigenvectors-eigenvalues-and-dimension-reduction/


\paragraph{Observation de la PCA pour les variables climatiques et de dégustations} Les différents types de variables, c'est-à-dire les données climatiques (températures, précipitations) ou les données de dégustations, on été réduits un par un sur 3 dimensions afin de pouvoir les visualiser sur un graphique et d'observer les différences entre les deux années disponibles. Ceci permet de se faire une idée des différences  

\begin{figure}[H]
	\includegraphics[scale=0.9]{PCA_Climat_All_1}
	\caption{\label{PCAClimatAll} PCA sur les données climatiques (Tmin, Tmax, Tmean, DTR et Prec) }
\end{figure}

\begin{figure}[H]
	\includegraphics[scale=0.9]{PCA_TMEAN_1}
	\caption{\label{PCAClimatTmean} PCA sur les températures moyennes de chaque mois }
\end{figure}

\begin{figure}[H]
	\includegraphics[scale=0.9]{PCA_PREC_1}
	\caption{\label{PCAClimatPrec} PCA sur les précipitations moyennes de chaque mois }
\end{figure}


Sur les figures \ref{PCAClimatAll}, \ref{PCAClimatTmean} et \ref{PCAClimatPrec}, on observe assez facilement deux groupes distincts se dessiner, un par année. On peut déjà en déduire que les deux années on été différentes sur le plan climatique. Afin de visualiser cette différence, nous pouvons observer la variation de la température maximale à Pereira, ce que nous montre la figure \ref{Tmax_Pereira}.

\begin{figure}[H]
	\includegraphics[scale=0.3]{Pereira_TMAX_2011_2016}
	\caption{\label{Tmax_Pereira} Variation des températures maximales à Pereira entre 2011 et 2016}
\end{figure}




\subsection{Clustering}
%SOM





\section{Prédiction}
Il a été possible ou non de faire de la prédiction, confusion matrix etc
ce que les méthodes comme random forest ont donné


\chapter{Discussion}
Analyse globale des résultats, discussion sur les résultats 
Données manquantes

Exemple: on sait que la fermentation du grain dans sa pulpe impacte sur la chimie de la graine et peut rendre le café plus ou moins acide -> pas de données sur la fermentation (temps, méthode etc) idem pour le séchage, le stockage, la récolte etc

	
	\chapter{Discussion des résultats}


%Analyse globale des résultats, discussion sur les résultats 
%Données manquantes

%Exemple: on sait que la fermentation du grain dans sa pulpe impacte sur la chimie de la graine et peut rendre le café plus ou moins acide -> pas de données sur la fermentation (temps, méthode etc) idem pour le séchage, le stockage, la récolte etc


% L'objectif aurait été de définir des gouts (tastewheel ) - pas de données

% Revenir sur les données 

% todo -> améliorer les modèles, changer les index (RMSE, RSquared etc) - nécessite théorie

% Portes qui s'ouvrent 

%TODO - Relire et relire, revenir surle rapport et relire, penser au résumé etc

Les différentes analyses réalisées, tant au niveau de l'apprentissage supervisé que non-supervisé n'ont pas permis de créer un modèle fiable de description ou de prédiction de la qualité du café. Cependant, certains résultats peuvent être mis en avant et certaines critiques peuvent être faites sur les données et les méthodes utilisées. \\

%Relation avec la pluie
\paragraph{Résultats exploitables} Concernant les résultats exploitables, une relation a été observée entre la quantité de pluie durant l'année et le nombre de défauts physiques du grain. Plus il y a de pluie, plus il y a de grains ayant des défauts et plus il y a de cafés ayant la note zéro (voir figure \ref{ThirdSOMASNM}). Rappelons que les grains défectueux sont éliminés par un processus industriel permettant l'élimination des grains ayant une couleur ou une densité anormale. Cependant, certains défauts, comme la présence de trous dûs aux parasites (Broca, ou Broca de punto), sont plus difficilement détectables et peuvent amener un goût désagréable au café. Un seul grain peut modifier le goût d'une tasse ! On remarque aussi que l'année 2011 a beaucoup moins de cafés dont la note dépasse les 80 (voir figure \ref{fig:pointtotauxetc}), score minimal pour avoir ma mention \textit{Specialty Coffee}. Il est donc raisonnable de penser que les fortes quantités de pluie, nuisent à la qualité du café et que plus il pleut, moins il y a de chance d'avoir de très bon cafés.  \\



% Manque de données
\paragraph{Données disponibles} Certain points concernant la qualité des données reçues sont à mettre en avant. Premièrement, les données de qualités de sols ne contenant que le pH, le taux de matières organiques et la texture, toutes les relations chimiques éventuelles entre les minéraux du sol et les arômes n'ont pas pu être explorées. La composition chimique du sol n'est pas la seule à avoir une influence possible sur la qualité et les saveurs du café. Les différents traitement du café une fois récolté, comme la fermentation ou le séchage, peuvent avoir une grande influence mais malheureusement aucune donnée n'a été fournie à ce sujet non plus. \\

\noindent Deuxièmement, les données de dégustation contenaient parfois, de manière difforme, des notes sur les arômes ou saveurs du café. Ces données peuvent être décrites comme sur la figure \ref{fig:coffeeflavorwheel} et pourraient permettre de décrire d'une manière plus sensorielle le goût du café et ainsi faire un éventuel lien avec les composants chimiques du sol et les pratiques culturales ou même le climat. Malheureusement, le manque de données en nombre d'une part et d'uniformité d'autre part n'ont pas permis d'intégrer ces informations au set de données et donc d'en analyser les éventuelles corrélations.\\

\begin{figure}[h]
	\centering
	\includegraphics[width=1\linewidth]{img/coffee_flavor_wheel}
	\caption{Roue des parfums du café}
	\label{fig:coffeeflavorwheel}
\end{figure}



%Customisation des méthodes, essais avec d'autres méthodes, traitement des variables catégorielles

%parler des données d'Eric sur la plante du café, de la taille etc



\noindent Il serait intéressant de prélever d'autres données, directement au niveau des fermes. Beaucoup de facteurs influencent la croissance d'une plante et la maturation de ses fruits en particulier les facteurs influençant la photosynthèse. Si une plante est particulièrement exposée à l'est, et donc au soleil du matin, la photosynthèse sera plus efficace car l'humidité ambiante est plus élevée, ce qui améliore les capacités d'assimilation de la lumière par la plante. Par contre, si la plante grandit sur un terrain ombragé, donc entourée d'autres plantes, l'humidité sera aussi plus élevée mais la plante devra fabriquer plus de feuilles pour capter plus de lumière et aura donc moins d'énergie à mettre dans les fruits. L'altitude peut aussi avoir une influence importante. Sur la luminosité, en particulier par rapport à la couverture nuageuse qui peut être parfois plus importante en montagne, mais aussi sur la température. Dans les lieux plus froids, la plante fonctionne plus lentement et les fruits ont besoin de plus de temps pour murir, ce qui laisse plus de temps aux arômes pour se développer et produit de manière générale des cafés de qualités.  \\



% si on veut prédire la qualité du café, il faut des données précises. Il peut pleuvoir des trombes sur une ferme, et 100m plus loin rien. 



%Résultat
\noindent Durant le projet, une réunion a été organisée avec Felipe Rincón, coordonnateur de gestion au comité départemental des caféiculteurs de Risaralda, qui est à l'origine des données sur le café que nous avons reçues. La possibilité m'a été donnée de poser des questions entre autre sur le processus de récolte et de centralisation des données, processus jusqu'alors inexistant. M. Rincón m'a alors confié que les différentes demandes effectuées pour ce travail ainsi que le résultat final de l'agglomération des données disponibles, l'ont poussé à commencer un processus de normalisation de la récolte des données. Il est donc envisageable de pouvoir retenter une étude de ce type lorsque les données existantes auront éventuellement été complétées et lorsque de nouvelles données, plus complètes et possédant moins de défauts, auront été générées. Parmi les défauts importants signalés à M. Rincón, on notera qu'aucune date de récolte n'a été fournie. Seul les dates de dégustation étaient disponibles et les périodes de croissance de la plante ont dû être estimées. \\


\paragraph{Méthodes utilisées} Il est aussi important de mentionner les méthodes utilisées. Random Forest, PLS, ou le clustering, peuvent être paramétrées de beaucoup de manières différentes et leur efficacité peut être testée et améliorée. Dans le cadre de ce projet, avec le peu d'heures à disposition, ces méthodes ont été utilisées de manière exploratoire afin d'avoir une première impression sur les possibilités de modélisation. D'un même point de vue, d'autres méthodes auraient pu être explorées en jouant sur les particularités de chacune d'entre-elles par rapport aux types de données en notre possession. Pour améliorer cette partie, un bagage théorique supplémentaire en statistique serait nécessaire afin de mieux cerner les subtilités des méthodes de modélisation.  \\

\noindent 


 














	
	

\chapter{Conclusion}
Qu'est-ce que le travail peut apporter, que faudrait-il faire pour qu'il apporte plus (données manquantes, autres méthodes à explorer) 


	
	\backmatter
	
	\appendix

\chapter{Description du dataset}





















\chapter{Importances des variables par cluster}
\label{annexe:clust}


\begin{table}[H]
	\centering
	\caption{HCPC avec trois clusters comparés à la sortie Acidez}
	\label{cluster3acidez}
	\begin{tabular}{llll}
		& 1  & 2  & 3   \\
		\hline
		5    & 0  & 3  & 0   \\
		5.5  & 0  & 1  & 0   \\
		6    & 3  & 6  & 18  \\
		6.25 & 1  & 1  & 3   \\
		6.5  & 5  & 8  & 35  \\
		6.75 & 1  & 0  & 7   \\
		7    & 39 & 50 & 102 \\
		7.25 & 13 & 13 & 24  \\
		7.5  & 53 & 40 & 76  \\
		7.75 & 1  & 6  & 8   \\
		8    & 0  & 59 & 47  \\
		8.25 & 0  & 8  & 4   \\
		8.5  & 0  & 1  & 0  
	\end{tabular}
\end{table}


\begin{table}[H]
	\centering
	\caption{Tableau des clusters pour la sortie Puntaje Total}
	\label{my-label}
	\begin{tabular}{llll}
		& 1  & 2  & 3  \\
		\hline
		42     & 0  & 0  & 1  \\
		49     & 0  & 1  & 2  \\
		50     & 0  & 3  & 0  \\
		52.5   & 0  & 0  & 1  \\
		58     & 0  & 0  & 1  \\
		58.5   & 0  & 1  & 0  \\
		59     & 0  & 2  & 0  \\
		59.5   & 0  & 1  & 0  \\
		60     & 0  & 3  & 2  \\
		60.5   & 0  & 2  & 0  \\
		61     & 1  & 0  & 1  \\
		61.75  & 0  & 1  & 0  \\
		62.5   & 0  & 0  & 1  \\
		63     & 0  & 1  & 0  \\
		64     & 0  & 0  & 1  \\
		65.5   & 0  & 0  & 1  \\
		66     & 0  & 1  & 1  \\
		67     & 0  & 1  & 4  \\
		68     & 1  & 0  & 2  \\
		68.75  & 0  & 1  & 0  \\
		69     & 0  & 1  & 1  \\
		69.5   & 0  & 0  & 1  \\
		69.75  & 0  & 0  & 1  \\
		70     & 0  & 0  & 1  \\
		70.75  & 1  & 0  & 0  \\
		71     & 1  & 2  & 3  \\
		71.375 & 0  & 0  & 1  \\
		71.75  & 0  & 1  & 0  \\
		72     & 0  & 1  & 2  \\
		72.5   & 0  & 1  & 0  \\
		72.75  & 0  & 0  & 1  \\
		73     & 0  & 5  & 3  \\
		73.25  & 0  & 1  & 0  \\
		73.5   & 0  & 0  & 2  \\
		74     & 0  & 0  & 2  \\
		74.75  & 0  & 1  & 1  \\
	\end{tabular}
	\begin{tabular}{llll}
		& 1  & 2  & 3  \\
		\hline
		74.875 & 0  & 0  & 1  \\
		75     & 1  & 2  & 2  \\
		75.25  & 1  & 0  & 4  \\
		75.375 & 0  & 0  & 1  \\
		75.5   & 1  & 3  & 2  \\
		75.75  & 0  & 1  & 2  \\
		76     & 2  & 2  & 8  \\
		76.125 & 0  & 0  & 2  \\
		76.25  & 0  & 0  & 2  \\
		76.375 & 1  & 0  & 0  \\
		76.5   & 4  & 3  & 6  \\
		76.75  & 1  & 0  & 4  \\
		77     & 4  & 3  & 8  \\
		77.25  & 0  & 1  & 2  \\
		77.5   & 1  & 0  & 3  \\
		77.75  & 1  & 2  & 5  \\
		77.875 & 1  & 0  & 0  \\
		78     & 7  & 4  & 3  \\
		78.25  & 0  & 1  & 1  \\
		78.5   & 1  & 1  & 7  \\
		78.75  & 0  & 0  & 2  \\
		79     & 13 & 10 & 41 \\
		79.25  & 1  & 3  & 5  \\
		79.45  & 0  & 0  & 1  \\
		79.5   & 0  & 2  & 4  \\
		79.52  & 0  & 0  & 1  \\
		79.625 & 0  & 0  & 2  \\
		79.75  & 0  & 1  & 1  \\
		80     & 6  & 7  & 8  \\
		80.25  & 0  & 1  & 4  \\
		80.375 & 0  & 0  & 1  \\
		80.5   & 2  & 1  & 8  \\
		80.625 & 1  & 0  & 0  \\
		80.75  & 1  & 2  & 8  \\
		81     & 4  & 0  & 6  \\
		81.25  & 1  & 3  & 6  \\
	\end{tabular}
	\begin{tabular}{llll}
		& 1  & 2  & 3  \\
		\hline
		81.375 & 1  & 0  & 1  \\
		81.5   & 3  & 3  & 8  \\
		81.625 & 1  & 0  & 1  \\
		81.75  & 1  & 1  & 9  \\
		82     & 7  & 4  & 13 \\
		82.25  & 7  & 2  & 4  \\
		82.375 & 2  & 0  & 2  \\
		82.5   & 29 & 24 & 19 \\
		82.75  & 5  & 3  & 4  \\
		83     & 0  & 10 & 14 \\
		83.25  & 0  & 3  & 5  \\
		83.5   & 1  & 7  & 6  \\
		83.75  & 0  & 3  & 3  \\
		84     & 0  & 5  & 8  \\
		84.25  & 0  & 5  & 2  \\
		84.5   & 0  & 8  & 5  \\
		84.75  & 0  & 6  & 1  \\
		85     & 0  & 7  & 6  \\
		85.25  & 0  & 4  & 0  \\
		85.5   & 0  & 1  & 2  \\
		85.75  & 0  & 3  & 4  \\
		86     & 0  & 4  & 1  \\
		86.25  & 0  & 0  & 1  \\
		86.5   & 0  & 3  & 2  \\
		86.75  & 0  & 1  & 3  \\
		87     & 0  & 1  & 2  \\
		87.25  & 0  & 2  & 0  \\
		87.5   & 0  & 1  & 0  \\
		87.75  & 0  & 1  & 0 \\
		&&&  \\
		&&&  \\
		&&&  \\
		&&&  \\
		&&&  \\
		&&&  \\
		&&&  
	\end{tabular}
	
\end{table}

\begin{table}[H]
	\centering
	\label{ClusterVarImp}
	\caption{Importance des variables lors de la réalisation des clusters}
	\begin{tabular}{lll}
		& Eta2       & P-value       \\
		\hline
		TmaxTotalAvg   & 0.79323282 & 8.808042e-216 \\
		TmaxTotal      & 0.79323282 & 8.808042e-216 \\
		tmean6         & 0.79308656 & 1.101237e-215 \\
		tmax6          & 0.77941800 & 6.561977e-207 \\
		TmeanTotalAvg  & 0.77802701 & 4.779121e-206 \\
		TmeanTotal     & 0.77802701 & 4.779121e-206 \\
		tmax7          & 0.77433881 & 8.706622e-204 \\
		tmean10        & 0.77341542 & 3.162373e-203 \\
		tmean5         & 0.77255468 & 1.047399e-202 \\
		tmean7         & 0.77146031 & 4.770308e-202 \\
		tmean9         & 0.76659843 & 3.682110e-199 \\
		tmax5          & 0.75947622 & 4.888995e-195 \\
		tmean8         & 0.75943990 & 5.127819e-195 \\
		tmax8          & 0.75506157 & 1.527633e-192 \\
		tmax9          & 0.73797359 & 2.717166e-183 \\
		tmin10         & 0.73685898 & 1.038321e-182 \\
		tmean4         & 0.72591249 & 4.041247e-177 \\
		tmax1          & 0.72445394 & 2.159972e-176 \\
		tmax4          & 0.72115391 & 9.274507e-175 \\
		tmax10         & 0.71764667 & 4.803959e-173 \\
		tmean1         & 0.71497543 & 9.398029e-172 \\
		tmin6          & 0.70461972 & 7.372790e-167 \\
		TminTotalAvg   & 0.68401310 & 1.306441e-157 \\
		TminTotal      & 0.68401310 & 1.306441e-157 \\
		tmean2         & 0.67984980 & 8.149567e-156 \\
		tmin8          & 0.67467087 & 1.293430e-153 \\
		tmax2          & 0.66718704 & 1.700396e-150 \\
		tmin9          & 0.66714064 & 1.776914e-150 \\
		tmean3         & 0.66366823 & 4.707493e-149 \\
		tmin7          & 0.65549751 & 9.209679e-146 \\
		tmax3          & 0.64170621 & 2.220623e-140 \\
		tmin5          & 0.63104980 & 2.317687e-136 \\
	\end{tabular}
\end{table}
\newpage
\begin{table}[H]
	\centering
	\label{ClusterVarImp2}
	\begin{tabular}{llll}
		& Eta2       & P-value       \\
		\hline
		tmin3          & 0.61739273 & 2.231189e-131 \\
		tmin4          & 0.61250703 & 1.225099e-129 \\
		tmin1          & 0.61199847 & 1.853488e-129 \\
		tmin2          & 0.60842700 & 3.343356e-128 \\
		DtrTotalAvg    & 0.55204696 & 9.200736e-110 \\
		DtrTotal       & 0.55204696 & 9.200736e-110 \\
		prec10         & 0.52838295 & 1.045148e-102 \\
		PrecTotalAvg   & 0.51318493 & 2.320750e-98  \\
		PrecTotal      & 0.51318493 & 2.320750e-98  \\
		ASNM           & 0.50548203 & 3.287540e-96  \\
		prec6          & 0.48690365 & 3.716337e-91  \\
		dtr8           & 0.47172566 & 3.664625e-87  \\
		prec4          & 0.46275569 & 7.423006e-85  \\
		dtr1           & 0.45506364 & 6.575620e-83  \\
		prec5          & 0.43486454 & 6.354576e-78  \\
		dtr7           & 0.43333703 & 1.488559e-77  \\
		dtr10          & 0.42305017 & 4.330758e-75  \\
		dtr2           & 0.41092218 & 3.056574e-72  \\
		dtr4           & 0.39912083 & 1.589035e-69  \\
		dtr5           & 0.39851482 & 2.183452e-69  \\
		dtr6           & 0.38517429 & 2.199730e-66  \\
		dtr9           & 0.38032942 & 2.611213e-65  \\
		dtr3           & 0.32744102 & 4.212873e-54  \\
		prec7          & 0.27997934 & 8.894746e-45  \\
		prec9          & 0.27175535 & 3.172161e-43  \\
		prec3          & 0.26360964 & 1.050396e-41  \\
		prec1          & 0.12139964 & 1.224259e-17  \\
		prec8          & 0.11382522 & 1.787707e-16  \\
		prec2          & 0.10476930 & 4.266965e-15  \\
		Luminosidad    & 0.03438122 & 6.133050e-05  \\
		Arcilloso      & 0.02674672 & 6.590146e-04  \\
		OrientationNum & 0.02251785 & 2.402284e-03  \\
		Limoso         & 0.01764046 & 1.041358e-02  \\
		pH\_avg        & 0.01665250 & 1.395976e-02  \\
		Cascajoso      & 0.01553506 & 1.940861e-02  \\
		Franco         & 0.01291747 & 4.161129e-02 
	\end{tabular}
\end{table}















\chapter{Partial Plots - résultats Random Forest}
























	
	\bibliography{mybib}{}
	\bibliographystyle{plain}
	
	\listoffigures
	
	\listoftables
	
	\printnomenclature
	
	
	
\end{document}