%\section*{Résumé}
%TODO résumé
%Résumé
\renewcommand{\abstractname}{Résumé}
\begin{abstract}
	
	
	Ce travail de bachelor présente une analyse des relations entre les données environnementales et les résultats de dégustations des cafés du département de Risaralda, en Colombie. Le but principal est de permettre au comité des caféiculteurs de Risaralda de caractériser leur cafés en relation avec les conditions de culture locale, à des fins principalement de marketing. Pour se faire, des données de dégustations on été rassemblées et associées aux conditions environnementales correspondantes afin de pouvoir y appliquer des méthodes de Machine Learning et d'en extraire des informations \\
	
	\noindent Dans une première partie, après un bref descriptif des méthodes utilisées pour ce projet, l'analyse a été réalisée de manière non-supervisée en utilisant les cartes auto-organisatrices, l'analyse en composante principale ou le clustering, afin de tenter de repérer des groupes différents les uns des autres et de tenter d'expliquer ces différences. En suite, des méthodes d'apprentissage supervisé comme \textit{Random Forest} ou \textit{Partial Least Squares} ont été utilisées afin d'explorer les capacités de prédiction avec les données à disposition. \\

	
	\noindent Les différents éléments obtenus ne permettent pas de prédire la qualité du café cependant des zones à risques peuvent être détectées et des défauts dans la collecte de données ont été mis en avant et sont en cours d'amélioration. 
	
	
\end{abstract}


